\documentclass[main]{subfile}

\begin{document}

\chapter{Reporte de error a Google}
\label{sec:google-error-report}
  
La autenticación de los clientes con el \gls{backend} se realiza mediante claves de sesión en el caso de la aplicación web, y con el protocolo OAuth~2.0 en el cliente Android (ver \cref{sec:arquitectura-autenticación}). Al estar este último protocolo estandarizado, suele ser conveniente utilizar librerías de terceros ya probadas con el fin de evitar errores.

Como el cliente Android es implementado en Java, la librería escogida fue la que Google pone a disposición del público, con el nombre \emph{Google OAuth Client Library for Java}~\autocite{google-oauth-client}. Esta librería contiene la clase \texttt{Credential}, que proporcionándole las \glspl{url} del llamado \emph{token endpoint}~\autocite[sec.~3.2]{oauthrfc} y los \glspl{token} de acceso y refresco iniciales, automatiza el refresco del \gls{token} de acceso cuando éste caduca, de forma transparente, tanto para el usuario como para el desarrollador.

Al utilizarlo, se comprobó que la librería no detectaba correctamente la caducidad del \gls{token}, identificando el mensaje de caducidad del servidor como un error de autenticación desconocido, y obligando al usuario a realizar el proceso de \emph{login} completo.

Tras un proceso intensivo de revisión del código de la librería y de \emph{debugging}, se encontró el error en el código de la librería. Resulta que el protocolo OAuth~2.0 indica \autocite[cap.~3]{oauth-bearer-rfc} que la invalidez de un \gls{token} se notifica indicando entre las cabeceras \gls{http} el texto \texttt{error="invalid_token"}.

La librería de Google realiza esta detección mediante la siguiente expresión regular: 

\mintinline{java}|INVALID_TOKEN_ERROR = Pattern.compile("\\s*error\\s*=\\s*invalid_token");|

Como se ve, esta expresión regular no tiene en cuenta las comillas del texto \texttt{invalid_token}, por lo que actua incorrectamente y no realiza el refresco del \gls{token}. El código correcto debería ser el siguiente:

\mintinline{java}|INVALID_TOKEN_ERROR = Pattern.compile("\\s*error\\s*=\\s*\"invalid_token\"");|

Este error, junto con el código propuesto para arreglar, se envió al sistema de soporte de la propia librería\footnote{\url{https://code.google.com/p/google-oauth-java-client/issues/detail?id=88}}, y actualmente, a la fecha de finalización de esta memoria, los desarrolladores de Google se encuentran investigándolo.

\end{document}
