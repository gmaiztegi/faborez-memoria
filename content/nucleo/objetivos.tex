 \documentclass[main]{subfiles}

\begin{document}

\section{Objetivos del proyecto}

% todo Terminar de definir objetivos del proyecto
Los objetivos del proyecto Faborez pueden describirse en los siguientes puntos:

\begin{enumerate}
  \item Desarrollar un servicio que de solución a los problemas planteados en el \cref{sec:antecedentes-problema}, mediante una aplicación para dispositivos móviles.
  \item Convertir a los usuarios de la aplicación en protagonistas de la aplicación, integrando sus valoraciones, críticas y comentarios en el desarrollo del proyecto.
  \item Además, de la interacción humana, poner en marcha los procedimientos técnicos que permitan fácilmente la detección de errores y su corrección.
  \item Implementar las características aportadas por los usuarios que se consideren viables, y recoger el resto de propuestas interesantes para que puedan ser implementadas en el futuro.
\end{enumerate}

Se excluyen del alcance del proyecto los puntos que a continuación se enumeran:

\begin{enumerate}
  \item El análisis de carácter legal que resultaría necesario en una aplicación final. La aplicación realiza una importante recogida de datos personales (mensajes enviados y geolocalización) que se almacenan en el servidor. Esta recogida requiere de una consideración en las distintas consecuencias legales, sobre todo los relativos a la \gls{lopd} y la \gls{lssi}, lo cual sería materia suficiente para un Proyecto Fin de Carrera separado.
  \item La labor de documentación y sistematización de la gestión del servidor de la aplicación. En el proyecto se pondrán en línea los sistemas esenciales para el funcionamiento de Faborez, pero no se profundizará en el análisis del rendimiento, disponibilidad o seguridad a nivel de sistema operativo.
  \item Como se ha mencionado en el \cref{sec:tech-android}, se utilizará Android como plataforma de desarrollo móvil y se descartarán el resto de sistemas operativos.
\end{enumerate}


\end{document}
