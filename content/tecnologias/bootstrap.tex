\documentclass[main]{subfiles}

\begin{document}

\section{Hoja de estilo: Bootstrap}
\label{sec:tech-bootstrap}

El desarrollo de cualquier aplicación web pasa por realizar un deseño de su interfaz en \gls{html}. Esta tarea generalmente se delega, cuando es posible, en un diseñador gráfico especializado, si se quiere obtener un diseño aceptable.

La calidad de la parte gráfica de la aplicación se encuentra fuera del alcance del proyecto, y al no contar con un diseñador que realice esta tarea, se ve necesario buscar alternativas que simplifiquen esta tarea. Para ello, existen disponibles en la red hojas de estilo \gls{css} prefabricadas y de licencia libre.

Estos diseños facilitan la maquetación del sitio web poniendo a la disposición del diseñador elementos comunes a la hora de diseñar: estructuras de columnas adaptables al ancho de la pantalla, formato de las tablas más atractivo que aquel por defecto de los navegadores, estilo de los textos con tipografías, diseño básico de menús de navegación, bloques de alerta y de notificación, 

De entre todas ellas, son dos las más utilizadas: \gls{bootstrap}~\autocite{bootstrap} y \gls{foundation}~\autocite{foundation}. La primera está desarrollada por Twitter, que lo utiliza como base para el diseño de su propia red social. \Gls{foundation}, por otra parte, es desarrollado por su empresa creadora, dedicada al diseño gráfico de portales web.

Entre estas dos alternativas, es \gls{bootstrap} con la cual se tenía mayor experiencia previa, por lo que se ha optado por esta tecnología con el único fin de reducir el tiempo de aprendizaje y desarrollo.

\end{document}
