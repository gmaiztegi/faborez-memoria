\documentclass[main]{subfiles}

\begin{document}

\chapter{Tecnología a desarrollar}
\label{sec:tech}

Un proyecto de Ingeniería Informática como este, se ha definido en gran medida por las distintas adquisiciones tecnológicas que se han ido incorporando al proyecto durante el desarrollo. Para cada una de ellas se explica la necesidad que motivó su adquisición, las alternativas existentes y las razones para la elección, y sus principales características.

Cada adquisición ha jugado un papel más o menos crucial en el proyecto: algunas de ellas forman la piedra angular del servicio y este hubiera resultado muy distinto con otra elección, como la base de datos o el \gls{framework} del \gls{backend}. Otros, en cambio, han jugado un papel menor y podrían ser reemplazables o incluso desechados, como la hoja de estilo.

En casi todas las tecnologías, factor un decisivo para la elección de ciertas alternativas frente a sus competidoras ha sido la experiencia previa con ellas, ya que se ha querido evitar grandes periodos de aprendizaje y los riesgos derivados.

Las secciones de este capítulo desarrollan las siguientes tecnologías: \gls{mongodb} como base de datos, \gls{sails} como \gls{framework} para el \gls{backend}, la plataforma de desarrollo móvil Android, implementación de lógica en el cliente web con \gls{backbone}, la hoja de estilo prefabricada \gls{bootstrap} y finalmente los servicios \gls{heroku} para el alojamiento y Bugsense para la recolección de reportes de error.

\subfile{content/tecnologias/mongodb}
\subfile{content/tecnologias/sails}
\subfile{content/tecnologias/android}
\subfile{content/tecnologias/backbone}
\subfile{content/tecnologias/bootstrap}
\subfile{content/tecnologias/heroku}
\subfile{content/tecnologias/bugsense}


\end{document}
