%Términos

\newglossaryentry{activity}
{
  name={\emph{Activity}},
  plural={\emph{Activities}},
  description={Clase de Android que implementa las interfaces de usuario y controla su ciclo de vida}
}

\newglossaryentry{apache}
{
  name={Apache},
  description={Servidor \gls{http} de código abierto}
}

\newglossaryentry{app}
{
  name={\emph{app}},
  description={Aplicación, generalmente para dispositivos móviles}
}

\newglossaryentry{aws}
{
  name={Amazon AWS},
  description={Proveedor de servicios \gls{iaas}. Web: \url{http://aws.amazon.com}}
}

\newglossaryentry{backbone}
{
  name={Backbone.js},
  description={Librería de JavaScript de estructuración de aplicaciones web}
}

\newglossaryentry{backend}
{
  name={\emph{backend}},
  description={Parte interna del \emph{software}, que procesa los datos según la lógica de la aplicación}
}

\newglossaryentry{bootstrap}
{
  name={Bootstrap},
  description={Hoja de estilo \gls{css} de licencia libre desarrollada por Twitter}
}

\newglossaryentry{bundle}
{
  name={\emph{bundle}},
  description={Módulo de \gls{symfony}, que da a este \gls{framework} funcionalidad añadida}
}

\newglossaryentry{bytecode}
{
  name={\emph{bytecode}},
  description={Código intermedio más abstracto que el código de máquina, que se ejecutará en una máquina virtual}
}

\newglossaryentry{cache}
{
  name={cache},
  description={Almacenamiento temporal de datos para la mejora del rendimiento}
}

\newglossaryentry{cookie}
{
  name={\emph{cookie}},
  description={Dato que un servidor web almacena en el navegador para su identificación en futuras peticiones}
}

\newglossaryentry{connect}
{
  name={Connect.js},
  description={\gls{framework} de \glspl{middleware} para \gls{node}}
}

\newglossaryentry{express}
{
  name={Express.js},
  description={\gls{microframework} basado en \gls{node} que implementa sobre éste la lógica de un servidor web}
}

\newglossaryentry{foundation}
{
  name={Foundation},
  description={Hoja de estilos \gls{css} desarrollada por ZURB}
}

\newglossaryentry{fragment}
{
  name={\emph{Fragment}},
  description={Interfaz secundaria de Android, que puede ser reutilizada en varios \glspl{activity}}
}

\newglossaryentry{framework}
{
  name={\emph{framework}},
  description={Estructura conceptual y tecnológica compuesta por herramientas, librerías, módulos, convenciones y una metodología de trabajo, que tiene como fin organizar y facilitar el desarrollo}
}

\newglossaryentry{frontend}
{
  name={frontend},
  description={Interfaz, parte del \emph{software} con la que interactúa el usuario}
}

\newglossaryentry{git}
{
  name={Git},
  description={\emph{Software} de control de versiones distribuido, desarrollado por Linus Torvalds}
}

\newglossaryentry{heroku}
{
  name={Heroku},
  description={Proveedor \gls{paas} basado en \gls{aws} que da soporte a distintas plataformas de desarrollo}
}

\newglossaryentry{hook}
{
  name={\emph{hook}},
  description={Método de interceptar llamadas a funciones o mensajes entre los componentes del \emph{software} para alterar su funcionamiento de la forma deseada}
}

\newglossaryentry{html5}
{
  name={HTML5},
  description={Quinta versión del formato \gls{html}, que incluye mejoras como la introducción directa de vídeos, audio y el acceso a la geolocalización mediante JavaScript, entre otros}
}

\newglossaryentry{jackson}
{
  name={Jackson},
  description={Librería de Java para la codificación y decodificación del formato \gls{json}}
}

\newglossaryentry{marionette}
{
  name={Marionette.js},
  description={Librería en JavaScript para ampliar la funcionalidad de \gls{backbone} tipos complejos de vista}
}

\newglossaryentry{maven}
{
  name={Maven},
  description={Herramienta de software para la gestión y construcción de proyectos}
}

\newglossaryentry{microframework}
{
  name={\emph{microframework}},
  description={\Gls{framework} relativamente ligero o de funcionalidades esenciales, comparado con tecnologías de la misma familia}
}

\newglossaryentry{middleware}
{
  name={\emph{middleware}},
  description={Módulo de \emph{software} intermediario entre dos o más elementos de la arquitectura}
}

\newglossaryentry{mongodb}
{
  name={MongoDB},
  description={Base de datos \gls{nosql} de almacenamiento de documentos}
}

\newglossaryentry{mysql}
{
  name={MySQL},
  description={Sistema de gestión de bases de datos \gls{sql}, de código libre}
}

\newglossaryentry{node}
{
  name={Node.js},
  description={Plataforma de desarrollo de servidores basado en JavaScript}
}

\newglossaryentry{nosql}
{
  name={NoSQL},
  description={Base de datos no relacional, que no utiliza
    el lenguaje \gls{sql}}
}

\newglossaryentry{play}
{
  name={Play! Framework},
  description={\Gls{framework} de desarrollo web basado en \gls{scala}}
}

\newglossaryentry{pull}
{
  name={\emph{pull}},
  description={Modelo de comunicación cliente--servidor, en el cual son los clientes los que llevan la iniciativa y solicitan los datos al servidor}
}

\newglossaryentry{push}
{
  name={\emph{push}},
  description={En una arquitectura cliente--servidor, el modelo en el cual el servidor por iniciativa propia envía un mensaje al cliente}
}

\newglossaryentry{redis}
{
  name={Redis},
  description={Motor de base de datos en memoria, que almacena los datos en forma de clave-valor}
}

\newglossaryentry{sails}
{
  name={Sails.js},
  description={\Gls{framework} de desarrollo de servicio web basado en \gls{node}, \gls{connect} y \gls{express}}
}

\newglossaryentry{symfony}
{
  name={Symfony},
  description={\Gls{framework} para desarrollo web basado en PHP}
}

\newglossaryentry{scala}
{
  name={Scala},
  description={Lenguaje de programación funcional y orientado a objetos}
}

\newglossaryentry{telegram}
{
  name={Telegram},
  description={Aplicación de mensajería instantánea para móviles, alternativa a WhatsApp}
}

\newglossaryentry{token}
{
  name={\emph{token}},
  description={Cadena de texto utilizada por un cliente de la aplicación para autenticarse ante el \gls{backend}}
}

\newglossaryentry{txapeldun}
{
  name={\emph{txapeldun}},
  plural={\emph{txapeldun}es},
  description={Campeón, representante de los usuarios de Faborez}
}

\newglossaryentry{webapp}
{
  name={\emph{webapp}},
  description={\Gls{app} desarrollada mediante \gls{html} y que, por tanto, es multiplataforma}
}

\newglossaryentry{websockets}
{
  name={WebSockets},
  description={Protocolo de comunicación bidireccional y \emph{full-duplex}}
}

\newglossaryentry{xhr}
{
  name={XMLHttpRequest},
  description={Clase JavaScript para realizar peticiones HTTP asíncronas sin recargar completamente la página que se muestra}
}

% Acrónimos

\newacronym{ajax}{AJAX}{\emph{Asynchronous JavaScript And XML}}
\newacronym{api}{API}{\emph{Application Programming Interface}}
\newacronym{css}{CSS}{\emph{Cascading Style Sheets}}
\newacronym[description={\emph{Cross Site Request Forgery}. Vulnerabilidad que aprovecha la confianza que un sitio tiene en el usuario}]{csrf}{CSRF}{\emph{Cross Site Request Forgery}}
\newacronym[description={\emph{Embedded JavaScript}. Librería libre de plantillas para JavaScript}]{ejs}{EJS}{\emph{Embedded JavaScript}}
\newacronym{gcm}{GCM}{\emph{Google Cloud Messaging}}
\newacronym{html}{HTML}{\emph{HyperText Markup Language}}
\newacronym{http}{HTTP}{\emph{Hypertext Transfer Protocol}}
\newacronym{iaas}{IaaS}{\emph{Infraestructure--as--a--Service}}
\newacronym[description={\emph{JavaScript Object Notation}. Formato de intercambio de datos alternativo a XML, basado en JavaScript}]{json}{JSON}{\emph{JavaScript Object Notation}}
\newacronym{lopd}{LOPD}{Ley Orgánica de Protección de Datos de Carácter Personal}
\newacronym{lssi}{LSSI}{Ley de Servicios en la Sociedad de la Información}
\newacronym{mvc}{MVC}{Modelo--Vista--Controlador}
\newacronym{mvp}{MVP}{\emph{Minimum Viable Product}}
\newacronym{paas}{PaaS}{\emph{Platform--as--a--Service}}
\newacronym[description={\emph{Plain Old Java Object}. Objetos Java simples, sin más atributos o métodos que los esenciales para trabajar con sus atributos}]{pojo}{POJO}{\emph{Plain Old Java Object}}
\newacronym[description={\emph{Relational State Transfer}. técnica de arquitectura \emph{software} para el desarrollo de sistemas web distribuidos}]{rest}{REST}{\emph{Relational State Transfer}}
\newacronym{saas}{SaaS}{\emph{Software--as--a--Service}}
\newacronym{sql}{SQL}{\emph{Structured Query Language}}
\newacronym{url}{URL}{\emph{Uniform Resource Locator}}
