\documentclass[main]{subfiles}

\begin{document}

\dedication{%
Gracias a mi director de proyecto, José Miguel Blanco Arbe, por hacer de \enquote{padre} durante todo el proyecto, no por fijarme los objetivos y los plazos, sino por obligarme a hacerlo yo mismo.

Gracias a mi padre y a mi madre, Roberto y Mertxe, por confiar ciegamente en mi trabajo en Donostia y por permitirme el lujo de estudiar la carrera con total independencia y libertad.

Gracias a todas y todos los \emph{Txapeldunes} de este proyecto que con sus aportaciones, \emph{feedback} e ideas han logrado que Faborez haya saltado la barrera de \enquote{aplicación para PFC} para lograr ser una aplicación real. A Goiatz, Uxue, Mikel, Unai y Alberto, gracias.

Gracias sobre todo a todos los que han luchado a mi lado por defender ideales sin interés para muchos. Primero, gracias a todos los que por casualidad he ido conociendo en la representación estudiantil: a todos los compañeros de RITSI (los cuales, sintiéndolo mucho, no podría terminar de enumerar), a Ugaitz, Jon Ander, Iker, Sandra, Jokin, Alex, Goiatz, Adrian y muchos más.

Y segundo, a los compañeros de Facultad con los que solamente con ilusión y ganas llevamos a cabo un proyecto llamado Magna SIS: Ander, Jon, Manex, Fernando, Jere y Alberto. Gracias también a toda la gente maravillosa del mundo de las Junior Empresas, que con mayor o menor afinidad, he ido conociendo.

No obstante, no es generosidad dar lo que a uno le sobra sino entregar lo que más necesita. Y es por ello que dedico un párrafo a agradecer a Joseba Egia, que habiendo participado en todo lo anterior (este PFC, la representación estudiantil y las Junior Empresas), me ha ayudado sin dudarlo cuando más lo he necesitado entregando el más preciado bien: el tiempo.

Gracias a todos aquellos que en mayor o menor medida me he topado en estos maravillosos seis años de universidad, sean estudiantes, profesores, investigadores, PAS, personal de cafetería, de limpieza o incluso \enquote{enemigos}.

Gracias a ti, lector, por tomarte la molestia de rescatar e interesarte por este proyecto y leer sus páginas.

\blockquote[Iñaki García García\cite{inakigarciaupvehu}]{A todos aquellos que antes que nosotros estuvieron, a todos los que habéis estado en este periodo, y a los que estáis por venir, \textbf{gracias}.}
}

\end{document}
